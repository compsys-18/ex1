\documentclass{jarticle}

\title{計算機システム演習 第一回レポート}
\author{17B13541 \and 細木隆豊}
\date{}

\begin{document}
  \maketitle

  \section{説明・工夫}
  課題1はモンテカルロ法の定義通りにプログラムを実装しました。ランダム関数は実行ごとに値を変える方法を実装し、testではサンプル数が$n=10^9$から出力までに時間がかかるので$n=10^8$まで100倍ごとに出力するようにしました。

  課題2はまずバブルソートで実装し、その後余裕があったのでクイックソート(基準値は中央値ではなく、配列の一番最初の要素ですので最適ではないです)を実装しました。testではサンプル数がでは配列のサイズが100のランダムな配列を用意しソートする前と後を出力するようにしました。
  \section{感想・質問等}
  2Qのアルゴリズムとデータ構造でポインタについて触れられていたが理解できておらず、今回の資料、課題で自分なりに理解できるところまで落とし込めたのでたいへんよかったと思う。
\end{document}
